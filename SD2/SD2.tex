\documentclass[12pt]{article}%
\usepackage{amsfonts}
\usepackage{fancyhdr}
\usepackage[hidelinks]{hyperref}
\usepackage[a4paper, top=2.5cm, bottom=2.5cm, left=2.2cm, right=2.2cm]%
{geometry}
\usepackage{times}
\usepackage{amsmath}
\usepackage{amsthm}
\usepackage{changepage}
\usepackage{amssymb}
\usepackage{graphicx}%
\setcounter{MaxMatrixCols}{30}
\newtheorem{theorem}{Theorem}
\newtheorem{corollary}[theorem]{Corollary}
\newtheorem{definition}[theorem]{Definition}
\newtheorem{lemma}[theorem]{Lemma}
\newtheorem{proposition}[theorem]{Proposition}
%\newenvironment{proof}[1][Proof]{\textbf{#1.} }{\ \rule{0.5em}{0.5em}}

\begin{document}

\title{COS 445 - Homework X, Problem Y} %Replace X with homework number, Y with problem number.
\author{Please don't write your name} %Write your name here
\date{\today}
\maketitle
\section*{Collaborators}
You are allowed to work with other students, visit office hours, consult external references, etc., but \textbf{must} write up solutions on your own.  Please reference the course infosheet (\url{http://www.cs.princeton.edu/~smattw/Teaching/infosheet445sp20.pdf}) for a complete description of the collaboration policy. List any students you collaborated with, along with any lecture notes/online sources/etc. you consulted with \textbf{in your collaboration statement. Do not list any collaborators here!}

\section*{Part A}

A good strategy would be to partition the blocks so that you can maximize the number of districts you win based on the percentage of blocks where you have the majority constituents. Very basically the area is broken into a group of blocks and all the blocks together is 100-percent of voters. Because there are only two districts, each district will have $\frac{N}{2}$ of the population.  Furthermore, this means that to win a district would require $ $\frac{N}{2} * \frac{1}{2} + 1 = \frac{N}{4}+1$ voters.  Additionally, the way that Alpha draws the districts will dictate what the districts will look like.  This is because once Beta picks the district it "approves" of, there would be only one more district needed and it would automatically be the other district that Alpha drew that Beta didn't pick. This means, alpha does not need to consider which district Beta would choose, and alpha should just draw the districts so that they maximize the number of districts they win.
\newline

There are three seperate cases that we must observe when $d = 2$:  Alpha has more total votes than Beta, less total votes than Beta, and Alpha has less than $\frac{1}{4}$ total votes.  
\newline

In the first case where Alpha has more total votes than Beta their strategy should be two draw two districts such that they have a majority in both.  In this case no matter which district Beta picks, Alpha will win both districts.  This works because Alpha will always have at-least two more votes than Beta and can distribute the votes such that it has a majority in both districts.  To be explicit if Alpha has a two vote majority they have $\frac{n}{2} + 2$ votes, which can be split into two groups of $\frac{n}{4} + 1$ voters.  Doing this would give them a majority in both districts.
\newline

In the case that Alpha has less total votes than Beta, Alpha's strategy should be to draw their districts such that they win in one district and lose in the other. In order to be able to win both districts, Alpha would have to have $> 25 \%$ in both districts which would sum to $> 50\%$ but we know Alpha has $ < 50\%$ of the total votes. However, to win one district, Alpha only needs $> 25\%$ of the votes. Therefore, if alpha has less total votes than Beta, but $> 25\%$ of the total votes, Alpha can win at most 1 district. Maximizing the districts alpha wins means drawing the districts so that they win a majority of blocks in one of the districts. Thus, they are guranteed to win one district.
\newline

In the case that Alpha has less than $\frac{1}{4}$ total votes there is no strategy that would enable Alpha to win a district. Winning one district would require $>$50-percent in one district, but in order to have $>$50-percent requires having at least $\frac{1}{4} + 1$ total votes. We know Alpha has $< 25\%$ of the total votes, therefore there is not strategy where they could win a district.
\newline

The PARTITION problem is NP-Hard.  Thus, what we are trying to do is hard in general because it relates to the PARTITION problem, in the PARTITION problem we are trying to split a multiset $S$ into two sub-sets $S_1$ and $S_2$ .  Take the case where Alpha has a slight majority over Beta and only has two more votes and $d=2$.  Hence Alpha has $\frac{n}{2} + 2$ and Beta has $\frac{n}{2} - 2$.  Alpha's strategy in this case should be to split its total votes such that it wins both districts.  The only possible way to do this is for Alpha to create two equivalent sub-sets with  $\frac{n}{4} + 1$ voters.  Therefore, this is a similar problem to PARTITION problem and must be at least as hard.

\section*{Part B}

When $d=3$, after Alpha initially proposes three districts, Beta should pick the district that maximizes their number of wins after picking that district.  To do this, Beta should freeze one district at a time and reason about how many of the remaining two districts they could win.  Whichever, "frozen" district allows Beta to maximize payoff, they should choose that district and they redistrict such that they win as many possible districts.  An optimal outcome for Beta besides winning all three districts would be two win two-out of three, which can happen if Beta chooses the first district such that in the remaining two they will have a majority.

Therefore, Alpha knowing that Beta will try to "freeze" each individual district in its initial proposal should attempt then Alpha should try to minimize the number of districts Beta can win based on its drawing.  There are four seperate cases that we must observe when $d=3$ for Alpha: Alpha has $<\frac{1}{6}$ total votes, Alpha has $>\frac{1}{6}$ and $<=\frac{1}{2}$ total votes, Alpha has $>\frac{1}{2}$ total votes.

In the case that Alpha has $<\frac{1}{6}$ total votes regardless of how they draw their districts, they cannot win any district because at least $\frac{1}{6} + 1$ total votes is required to win a district.
In the case that Alpha has $>\frac{1}{6}$ and $<=\frac{1}{3}$ total votes Alpha should draw the three districts such that in two of the districts they lose and that in one of the districts they win.  In this case the worst that Alpha can do is tie in a district if they have $\frac{1}{6}$ total votes (assuming Beta picks correct district to freeze), but if they have $<=\frac{1}{3}$ total votes they will be guranteed to win one district.  This is the maximum districts that Alpha could win with $<\frac{1}{2}$ total votes.
In the case that Alpha has $>\frac{1}{2}$ total votes they should draw the districts such that they win two districts and lose one district, unless they have enough voters to draw it such that they win all districts.  In this case, no matter which decision Beta makes, Alpha is guranteed to win two districts (assuming that Beta doesn't draw the districts in such a way that there is a tie, which would detrimental to both and mean that Beta wouldn't win any districts).  

\end{document}